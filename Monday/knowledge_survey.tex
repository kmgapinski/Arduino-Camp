\documentclass{exam}
\begin{document}


\hfill\makebox[1in][r]{\vspace{.1in}Score: \rule{.5in}{1pt}}
\begin{center}
  \fbox{\fbox{\centering
    {\Large Arduino Camp Pre \& Post Knowledge Survey}.}}
\end{center}
\vspace{.2in}
\makebox[\textwidth]{Name: \enspace\hrulefill}

\begin{questions}
 
  \question Arduino is a(n) \rule{1in}{1pt}.
    \begin{choices}
      \choice micro-processor
      \choice full computer
      \choice micro-controller
      \choice smart phone
    \end{choices}
  
  \question An LED allows electricity to flow \rule{1in}{1pt}.
    \begin{choices}
      \choice 1-way.
      \choice 2-ways.
      \choice 4-ways.
      \choice does not allow electricity to flow at all. 
    \end{choices}
  
  \question Between +5V and GND (0V) an LED should have a(n) \rule{1in}{1pt}.
    \begin{choices}
      \choice motor
      \choice resistor
      \choice transistor
      \choice digitalWrite()
    \end{choices}
  
  \question In Arduino Code (C) the correct way to declare and set an "int" variable to 2 is \rule{1in}{1pt}.
    \begin{choices}
      \choice float 2 = x;
      \choice int x <= 2; 
      \choice set int x to 2;
      \choice int x = 2;
    \end{choices}
  
  \question The tone of the buzzer can be changed by changing the \rule{1in}{1pt} of the signal. 
    \begin{choices}
      \choice Frequency (Hz)
      \choice analogRead()
      \choice LED
      \choice password
    \end{choices}
  
  \question What will NOT dim an LED? 
    \begin{choices}
      \choice PWM (duty cycle).
      \choice photo resistor.
      \choice resistor.
      \choice amplifier. 
    \end{choices}
  
  \question What command will turn ON the LED on pin 13? 
    \begin{choices}
      \choice digitalRead(13); 
      \choice digitalWrite(13, LOW);
      \choice digitalWrite(13, HIGH);
      \choice pinMode(13, OUTPUT)
    \end{choices}
  
  \question How many times will this command loop before exiting?
  \fbox{ for(index = 0; index <= 7; index++)}
    \begin{choices}
      \choice 0 
      \choice 6
      \choice 7
      \choice 8
    \end{choices}
  
  
  \end{questions}
\end{document}
